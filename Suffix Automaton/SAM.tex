\documentclass[11pt]{report}

\usepackage{fullpage}
\usepackage[margin=1in]{geometry}
\usepackage{algorithm}
\usepackage{algorithmicx}
\usepackage[noend]{algpseudocode}
\usepackage{eqparbox}
\usepackage{amsmath}
\usepackage{amsthm}
\usepackage{amsfonts}
\usepackage{enumerate}
\usepackage{verbatim}
\usepackage{pgf}
\usepackage{tikz}
%\usepackage{parskip}
\usetikzlibrary{arrows,automata,positioning}
\usepackage[latin1]{inputenc}

\renewcommand{\thealgorithm}{}
\renewcommand{\algorithmicrequire}{\textbf{Input:}}
\renewcommand{\algorithmicensure}{\textbf{Output:}}
\renewcommand{\algorithmicrepeat}{\textbf{Do}}
\renewcommand{\algorithmicuntil}{\textbf{While}}
\renewcommand\algorithmiccomment[1]{%
  \hfill\ \eqparbox{COMMENT}{#1}%
}

\newtheorem{theorem}{Theorem}[chapter]
\newtheorem{lemma}[theorem]{Lemma}
\newtheorem{proposition}[theorem]{Proposition}
\newtheorem{corollary}[theorem]{Corollary}
\newtheorem{definition}[theorem]{Definition}

\begin{document}
\title{An Introduction to Suffix Automaton\\(Survey Report)}
\author{Shimi Zhang}
\date{Nov. 2, 2013}
\maketitle

\chapter{Prerequisites}

Some definitions:
\begin{enumerate}
\item $\varepsilon$ = an empty word.
\item $\mathcal{A}$ = a finite set called the \textit{alphabet}.
\item $\mathcal{A}^*$ = the set of words over $\mathcal{A}$ and
  $\varepsilon$.
\item $\mathcal{A}^+ = \mathcal{A}^* - \left\{\varepsilon\right\}$
\item $|w|$ = the length of word $w$.
\item Suff$(y)$ = a set of all suffixes of word $y$.
\item Fact$(y)$ = a set of all factors (substring) of word $y$. 
\end{enumerate}
The suffix automaton is denoted by $\mathfrak{A}(y)$. Its states are
classes of the (right) syntactic equivalence associated with
Suff\((y)\), that is, are the sets of factors of \(y\) having the same
right context within \(y\). This states are in one-to-one
correspondence with the (right) contexts of the factors of \(y\) in
\(y\) itself. Formally, we call the (right) context of a word \(u\) is
\(\mathcal{R}(u)=u^{-1}\text{Suff}(y)\). 

We denote by \(\equiv_y\) the syntactic congruence which is defined, for \(u,v \in \mathcal{A}^*\), by$$u \equiv_y v$$ if and only if $$\mathcal{R}_y(u) = \mathcal{R}_y(v).$$ One can also identify the states of \(\mathfrak{A}(y)\) to sets of indices on y which are end positions of occurrences of equivalent factors. For any factor \(u\) of \(y\), one denotes by $$\textit{end-pos}(u) = \min\left\{|wu|\text{ }|\text{ }wu \text{ is a prefix of } y\right\} - 1,$$ the right position of the first occurrence of \(u\) in \(y\). Note that \(\textit{end-pos}(\varepsilon)=-1\).

\begin{lemma}
  Let \(u, v \in \text{Fact}(y) \text{ with } |u| \le |v|\). Then, $$u \textit{ is a suffix of } v \textit{ implies } \mathcal{R}_y(v) \subseteq \mathcal{R}_y(u)$$ and $$\mathcal{R}_y(u) =\mathcal{R}_y(v) \textit{ implies both end-pos}(u)=\textit{end-pos}(v) \textit{ and } u \textit{ is a suffix of } v.$$
\end{lemma}
\begin{proof}
  Let us suppose that $u$ is a suffix of $v$, Let $z \in
  \mathcal{R}_y(v)$. By definition, $vz$ is a suffix of $y$ and, since
  $u$ is a suffix of $v$, the word $uz$ is also a suffix of $y$. Thus,
  $z \in \mathcal{R}_y(u)$, which proves the first implication.\\
  Let us now suppose $\mathcal{R}_y(u) = \mathcal{R}_y(v)$. let $w, z$
  be such that $y = w \cdot z$ with $|w| = \textit{end-pos}(u) +
  1$. By definition of \textit{end-pos}, $u$ is a suffix of
  $w$. Therefore, $z$ is the longest word in $\mathcal{R}_y(v)$, which
  yields $|w| = \textit{end-pos}(v)+1$. The words $u$ and $v$ are thus
  both suffixes of $w$, and as $u$ is shorter than $v$ one obtains
  that $u$ is a suffix of $v$. This finishes the proof of the second
  implication and the whole proof.
\end{proof}



\begin{lemma}
  Let \(u, v, w \in \text{Fact}(y)\). If \(u\) is a suffix of \(v\),
  \(v\) is a suffix of \(w\) and \(u \equiv_y w\), then \(u \equiv_y v
  \equiv_y w, \).
\end{lemma}
\begin{proof}
  By Lemma 1.1, the assumption implies $$\mathcal{R}_y(w) \subseteq
  \mathcal{R}_y(v) \subseteq \mathcal{R}_y(u).$$ Then, the equivalence
  $u \equiv_y w$ which means $\mathcal{R}_y(u) = \mathcal{R}_y(w)$
  gives the conclusion.
\end{proof}

\begin{corollary}
Let \(u, v\in \mathcal{A}^*\). Then, the contexts of \(u\) and \(v\) are comparable for inclusion or are disjoint, that is, at least one of the three following conditions is satisfied:
\begin{enumerate}
\item \(\mathcal{R}_y(u) \subseteq \mathcal{R}_y(v)\),
\item \(\mathcal{R}_y(v) \subseteq \mathcal{R}_y(u)\),
\item \(\mathcal{R}_y(u) \cap \mathcal{R}_y(v) = \emptyset\).
\end{enumerate}
\end{corollary}
\begin{proof}
  One proves the property by showing that the
  condition $$\mathcal{R}_y(u) \cup \mathcal{R}_y(v) \not=
  \emptyset$$implies$$\mathcal{R}_y(u) \subseteq \mathcal{R}_y(v)
  \textit{  or  }
  \mathcal{R}_y(v) \subseteq \mathcal{R}_y(u).$$ Let $z \in
  \mathcal{R}_y(u) \cup \mathcal{R}_y(v)$. Then, $uz, vz$ are suffixes
  of $y$, and $u, v$ are suffixes of $yz^{-1}$. Consequently, among
  $u$ and $v$ one is a suffix of the other. One obtains finally the
  conclusion by Lemma 1.1.
\end{proof}
\noindent \textbf{Suffix function}\\
On the set Fact\((y)\) we consider the function \(s_y\) called the
\(\textit{suffix function}\) of \(y\). It is defined, for all \(v \in
\text{Fact}(y)\backslash\left\{ \varepsilon\right\}\), by$$s_y(v) =
\text{ longest suffix } u \text{ of } v \text{ such that } u
\not\equiv_y v.$$ After Lemma 1.1, one deduces the equivalent
definition:$$s_y(v) = \text{longest suffix }u \text{ of } v \text{
  such that } \mathcal{R}_y(v) \subset \mathcal{R}_y(u).$$ Note that,
by definition, $s_y(v)$ is a proper suffix of $v$, i.e., $|s_y(v)| <
|v|$. The following lemma shows that the suffix function $s_y$ induces
a failure function on state of $\mathfrak{A}(y)$.

\begin{lemma}
  Let \(u, v \in \text{Fact}(y)\backslash\left\{\varepsilon\right\}\). If \(u \equiv_y v\), then \(s_y(u) = s_y(v)\).
\end{lemma}
\begin{proof}
  By Lemma 1.1 one can suppose without loss of generality that $|u|
  \le |v|$, so $u$ is
  a suffix of $v$. The word $u$ cannot be a suffix of $s_y(v)$ because
  Lemma 1.2 would imply $s_y(v) \equiv_y v$, which contradicts the
  definition of $s_y(v)$. Consequently, $s_y(v)$ is a suffix of
  $u$. Since, by definition, $s_y(v)$ is the longest suffix of $v$
  which is not equivalent to itself, it is also $s_y(u)$. Thus, $s_y(u)=s_y(v)$.
\end{proof}
\begin{lemma}
  Let $y \in \mathcal{A}^+$. The word $s_y(y)$ is the longest suffix
  of $y$ that appears at least twice in $y$ itself.
\end{lemma}
\begin{proof}
  The context $\mathcal{R}_y(y)$ is $\left\{\varepsilon\right\}$. As
  $y$ and $s_y(y)$ are not equivalent, $\mathcal{R}_y(s_y(y))$
  contains some nonempty word $z$. Then $s_y(y)z$ and $s_y(y)$ are
  suffixes of y, which shows that $s_y(y)$ appears twices at least in
  $y$. Any suffix $w$ of $y$, longer than $s_y(y)$, is equivalent to
  $y$ by definition of $s_y(y)$. It thus satisfies $\mathcal{R}_y(w) =
  \mathcal{R}_y(y) = \left\{\varepsilon\right\}$. Which shows that $w$
  appears only once in $y$ and finishes the proof.
\end{proof}
\begin{lemma}
  Let $u \in
  \text{Fact}(y)\backslash\left\{\varepsilon\right\}$. Then, any word
  equivalent to $s_y(u)$ is a suffix of $s_y(u)$.
\end{lemma}
\begin{proof}
  Let $w = s_y(u)$ and $v \equiv_y w$. We show that $v$ is a suffix of
  $w$. The word $w$ is a proper suffix of $u$. If the conclusion of
  the statement is false, according to Lemma 1.1 one obtains that $w$
  is a proper suffix of $v$. Then let $z \in \mathcal{R}_y(u)$. As $w$
  is a suffix of $u$ equivalent to $v$, we have $z \in
  \mathcal{R}_y(w) = \mathcal{R}_y(v)$. Then, $u$ and $v$ are both
  suffixes of $yz^{-1}$, which implies that one is a suffix of the
  other. But this contradicts either the definition of $w = s_y(u)$ or
  the conclusion of Lemma 1.2, and proves that $v$ is a suffix of $w =
  s_y(u)$.
\end{proof}
The on-line construction of suffix automaton relies on the
relationship between $\equiv_{wa}$ and $\equiv_w$ which we examine
here. By doing this, we consider that the generic word $y$ is equal to
$wa$ for some letter $a$. The properties detailed below are also used
to derie precise bounds on the size of the automaton.
\begin{lemma}
  Let $w \in \mathcal{A}^*$ and $a \in \mathcal{A}$. The congruence
  $\equiv_{wa}$ is a refinement of $\equiv_w$, that is for all words
  $u, v \in \mathcal{A}^*$, $u \equiv_{wa} v$ implies $u \equiv_w v$.
\end{lemma}
\begin{proof}
  Let us assume that $u \equiv_{wa} v$, that is, $\mathcal{R}_{wa}(u)
  = \mathcal{R}_{wa}(v)$, and show that $u \equiv_w v$, that is, $\mathcal{R}_{w}(u)
  = \mathcal{R}_{w}(v)$. We show $\mathcal{R}_{w}(u) \subseteq
  \mathcal{R}_{w}(v)$ only because the opposite inclusion results by
  symmetry.\\
If $\mathcal{R}_w(u) = \emptyset$ the inclusion is clear. If not, let
$z \in \mathcal{R}_w(u)$. Then $uz$ is a suffix of $w$, which implies
that $uza$ is a suffix of $wa$. The assumption gives that $vza$ is a
suffix of $wa$, and thus $vz$ is a suffix of $w$, or $z \in
\mathcal{R}_w(v)$, which finishes the proof.
\end{proof}
\begin{lemma}
  Let $w \in \mathcal{A}^*$ and $a \in \mathcal{A}$. Let $z$ be the
  longest suffix of $wa$ that appears in $w$. If $u$ is a suffix of
  $wa$ strictly longer than $z$, then the equivalence $u \equiv_{wa}
  wa$ holds.
\end{lemma}
\begin{proof}
  It is a direct consequence of Lemma 1.5 because $z$ occurs at least
  twice in $wa$.
\end{proof}
Before going to the main theorem we state an additional relation
concerning right contexts.
\begin{lemma}
  Let $w \in \mathcal{A}^*$ and $a \in \mathcal{A}$. Then, for each
  word $u \in \mathcal{A}^*$,
  \begin{equation*}
    \mathcal{R}_{wa}(u) = \left\{
      \begin{aligned}
        &\left\{\varepsilon\right\} \cup \mathcal{R}_w(u)a& &\text{if
          $u$ is a suffix of } wa,&\\
        &\mathcal{R}_w(u)a& &otherwise.&
      \end{aligned}
    \right.
  \end{equation*}
\end{lemma}
\begin{proof}
  First notice that $\varepsilon \in \mathcal{R}_{wa}(u)$ is
  equivalent to: $u$ is a suffix of $wa$. It is thus enough to show
  $\mathcal{R}_{wa}(u)\backslash\left\{\varepsilon\right\}=\mathcal{R}_w(u)a$. \\
  Let $z$ be a nonempty word of $\mathcal{R}_{wa}(u)$. We get that $uz$
  is a suffix of $wa$. The word $uz$ can be written $uz'a$ with $uz'$
  a suffix of $w$. Consequently, $z' \in \mathcal{R}_w(u)$ and thus $z
  \in \mathcal{R}_{w}(u)a$.\\
  Conversely, let $z$ be a (nonempty) word in $\mathcal{R}_w(u)a$. It
  can be written $z'a$ for $z' \in \mathcal{R}_w(u)$. Thus, $uz'$ is a
  suffix of $w$, which implies that $uz = uz'a$ is a suffix of $wa$,
  that is, $z \in \mathcal{R}_{wa}(u)$. This proves the converse
  statement and ends the proof.
\end{proof}
\begin{theorem}
  Let $w \in \mathcal{A}^*$ and $a \in \mathcal{A}$. Let $z$ be the
  longest suffix of $wa$ that appears in $w$. Let $z'$ be the longest
  factor of $w$ for which $z' \equiv_w z$. Then for each $u, v \in
  \text{Fact}(w)$, $$u \equiv_w v \text{ and } u \not\equiv_w z \text{
  imply } u \equiv_{wa} v.$$ Moreover, for each word $u$ such as $u
\equiv_w z$,
\begin{equation*}
  u \equiv_{wa} \left\{
      \begin{aligned}
        &z& &if|u| \le |z|,&\\
        &z'& &otherwise.&
      \end{aligned}
    \right.
\end{equation*}
\end{theorem}
\begin{proof}
  Let $u, v \in \text{Fact}(w)$ be such that $u \equiv_w v$. By
  definition of the equivalence we get $\mathcal{R}_w(u) =
  \mathcal{R}_w(v)$. We suppose first that $u \not\equiv_w z$ and show
  that $\mathcal{R}_{wa}(u) = \mathcal{R}_{wa}(v)$, which is $u
  \equiv_{wa} v$.\\
According to Lemma 1.9, we have just to show that $u$ is a suffix of
$wa$ if and only if $v$ is a suffix of $wa$. Indeed, it is enough to
show that if $u$ is a suffix of $wa$ then $v$ is a suffix of $wa$
since the opposite implication results by symmetry.\\
So let us suppose that $u$ is a suffix of $wa$. We deduce from the
fact that $u$ is a factor of $w$ and the definition of $z$ that $u$ is
a suffix of $z$. We can thus consider the greatest integer $j \ge 0$
for which $|u| \le |s_{w}^j(z)|$. Let us note that $s_{w}^j(z)$ is a
suffix of $wa$ (like $z$ is), and the Lemma 1.2 ensures that $u
\equiv_w s_{w}^j(z)$. From which we get $v \equiv_w s_{w}^j(z)$ by
transitivity.\\
Since $u \not\equiv_w z$, we have $j > 0$. Lemma 1.6 implies that $v$
is a suffix of $s_{w}^j(z)$, and thus also of $wa$ as wished. This
proves the first part of the statement.\\
Let us consider now a word $u$ such as $u \equiv_w z$.\\
When $|u| \le |z|$, to show $u \equiv_w z$ by using the above
argument, we have only to check that $u$ is a suffix of $wa$ because
$z$ is a suffix of $wa$. This, in fact, is a simple consequence of
Lemma 1.1.\\
Let us suppose $|u| > |z|$. The existence of such a word $u$ implies
$z' \not= z$ and $|z'| > |z|$ ($z$ is a proper suffix of
$z'$). Consequently, by the definition of $z$, $u$ and $z'$ are not
suffixes of $wa$. Using the above argument again, this proves $u
\equiv_{wa} z'$ and finishes the proof.
\end{proof}
\begin{corollary}
  Let $w \in \mathcal{A}^*$ and $a \in \mathcal{A}$. Let $z$ be the
  longest suffix of $wa$ that appears in $w$. Let $z'$ be the longest
  word such as $z' \equiv_w z$. Let us suppose $z' = z$. Then, for
  each $u, v \in \text{Fact}(w)$, $$u \equiv_w v \text{ implies } u
  \equiv_{wa} v.$$
\end{corollary}
\begin{proof}
  Let $u, v \in \text{Fact}(w)$ be such that $u \equiv_w v$. We prove
  the equivalence $u \equiv_{wa} v$. The conclusion comes directly
  from Theorem 1.10 if $u \not\equiv_w z$. Else, $u \equiv_{wa} z$; by
  the assumption made on $z$ and Lemma 3.1, we get $|u| \le
  |z|$. Finally, Theorem 1.10 gives the same conclusion.
\end{proof}
\begin{corollary}
  Let $w \in \mathcal{A}^*$ and $a \in \mathcal{A}$. If the letter $a$
  does not appear in $w$, then, for each $u, v \in
  \text{Fact}(w)$, $$u \equiv_w v \text{ implies } u \equiv_{wa} v.$$
\end{corollary}
\begin{proof}
  Since $a$ does not appear in $w$ the word $z$ of Corollary 1.11 is
  the empty word. It is of course the longest of its class, which
  makes it possible to apply Corollary 1.11 and gives the same conclusion.
\end{proof}
\chapter{Suffix automaton}
The \textit{suffix automaton} of a word $y$ is the minimal automaton
that accepts the set of suffixes of $y$. It is denoted by
$\mathfrak{A}(y)$. The most surprising property of the automaton is
that its size is linear in the length of $y$ although the number of
factor of $y$ can be quadratic. The construction of the automaton also
takes a linear time on a fixed alphabet.
\section{Size of suffix automata}
The size of an automaton is expressed both by the number of its states
and the number of its edges. We show that $\mathfrak{A}(y)$ has less
than $2|y|$ states and less than $3|y|$ edges, for a total size
$O(|y|)$. This result is based on Theorem 1.10.
\begin{proposition}
  Let $y \in \mathcal{A}^*$ be a word of length n and let $st(y)$ be
  the number of states of $\mathfrak{y}$. For $n = 0$, we have $st(y)
  = 1$; for $n = 1$, we have $st(y) = 2$; for $n > 1$ finally, we
  have$$n + 1 \le st(y) \le 2n - 1,$$ and the upper bound is reached
  if and only if $y$ is of the form $ab^{n-1}$, for two distinct
  letters $a, b$.
\end{proposition}
\begin{proof}
  The equalities concerning short words can be checked direclty
  including $st(y) = 3$ when $|y| = 2$. Let us suppose $n > 2$. The
  minimal number of states of $\mathfrak{A}$ is obviously $n+1$
  (otherwise the path labelled by $y$ would contain a cycle yielding
  an infinite number of words recognized by the automaton); the
  minimal value is reached with $y = a^n (a \in \mathcal{A})$.\\
Let us show the upper bound. By Theorem 1.10, each letter $y[i], 2 \le
i \le n - 1$, increases by at most two the number of states of
$\mathfrak{A}(y[0..i-1])$. As the number of states of
$\mathfrak{A}(y[0]y[1])$ is $3$, it follows that
\begin{equation*}
  \begin{aligned}
    st(y) &\le& &3 + 2(n - 2)&\\
          &=& &2n-1,&
  \end{aligned}
\end{equation*}
as announced.\\
The construction of a word of length $n$ whose suffix automaton has
$2n-1$ states is still a simple application of Theorem 1.10 by noting
that each letter $y[2],y[3],...,y[n-1]$ must effectively lead to the
creation of two states during the construction. Notice that after the
choice of the first two letters, which must be different, there is no
choice for the other letters. This produces the only possible form
given in the statement.
\end{proof}
\begin{lemma}
  Let $y \in \mathcal{A}^+$ and let $ed(y)$ be the number of edges of
  $\mathfrak{A}(y)$. Then, $$ed(y) \le st(y) + |y| - 2$$.
\end{lemma}
\begin{proof}
  Let us call $q_0$ the initial state of $\mathfrak{A}(y)$, and
  consider the spanning tree of longest paths starting at $q_0$ in
  $\mathfrak{A}(y)$. The tree contains $st(y) - 1$ edges of
  $\mathfrak{A}(y)$ because it arrives exactly at one edge on each
  state except on the initial state.\\
  With each other edge$(p,a,q)$ we associate the suffix $uav$ of $y$
  defined as follows: $u$ is the label of the path starting at $q_0$
  and ending at $p$; $v$ is the label of the longest path from $q$
  arriving on a terminal state. Doing so, we get an injection from the
  set of concerned edges to the set Suff$(y)$. The suffixes $y$ and
  $\varepsilon$ are not concerned because they are labels of paths in
  the spanning tree. This shows that there is at most
  Card(Suff($y)\backslash \left\{y, \varepsilon\right\})=|y|-1$
  additional edges. Summing up the numbers of edges of the two types,
  we get at maxium of $st(y) + |y| - 2$ edges in $\mathfrak{A}(y)$.
\end{proof}
\section{Suffix links and suffix paths}
Theorem 1.10 and its two consecutive corollaries provide the frame of
the on-line construction of the suffix automaton
$\mathfrak{A}(y)$. The algorithm controls the conditions which appear
in these statements by means of function defined on the states of the
automaton, the suffix link function, and of a classification of the
edges into solid and non-solid edges. We define these two concepts
hereafter.
Let $p$ be a state of $\mathfrak{A}(y)$, different from the inital
state. State $p$ is a class of factors of $y$ that are equivalent with
respect to equivalence $\equiv_y$. Let $u$ be any word in the class
($u \not= \varepsilon$ because p is not the initial state). We define
the suffix link of $p$, denoted by $f_y(p)$, as the congruence class
of $s_y(u)$. The function $f_y$ is called the suffix link function of
the automaton. According to Lemma 1.4 the value of $s_y(u)$ is
independent of the word $u$ chosen in the class of $p$, which makes
the definition coherent. The suffix link function is also called a
failure function and used with this meaning in pattern matching
machine.
For a state $p$ of $\mathfrak{A}(y)$, we denote by $lg_y(p)$ the
maxmimum length of words $u$ in the congruence class of $p$. It is
also the length of the longest path starting from the initial state
and ending at $p$. The longest paths starting at the initial state
form a spanning tree for $\mathfrak{A}(y)$ (a consequence of Lemma
1.1). Edges which belong to this tree are qualified as
\textit{solid}. In an equivalent way, $$edge(p, a, q) \text{ is
  solid}$$ if and only if $$lg_y(q) = lg_y(p) + 1.$$ This notion is
used in the construction of the automaton.
Suffix links induce by iteration what we call suffix paths in
$\mathfrak{A}(y)$. One can note that $$q = f_y(p) implies lg_y(q) <
lg_y(p).$$ So, the sequence $$\langle p, f_y(p),f_y^2(p),... \rangle$$
is finite and ends at the initial state (which does not have a suffix
link). It is called the suffix path of $p$ in $\mathfrak{A}(y)$, and
is denoted by $\textit{SP}(p)$.
Let \textit{last} be the state of $\mathfrak{A}(y)$ that is the class
of word $y$ itself. This state is characterized by the fact that it is
not the origin of any edge. The suffix path of \textit{last} plays an
important part in the on-line construction. It is used to effectively
test conditions of Theorem 1.10 and its corollaries.\\
We denote by $\delta$ the transition function of $\mathfrak{A}(y)$.
\begin{proposition}
  Let $u \in \text{Fact}(y)\backslash\left\{\varepsilon\right\}$ and
  let $p = \delta(q_0,u)$. Then, for each integer $j \ge 0$ for which
  $s_y^j(u)$ is defined, $$f_y^j(p) = \delta(q_0,s_y^j(u)).$$
\end{proposition}
\begin{proof}
  We prove the result by recurrence on $j$. If $j = 0$, $f_y^j(p) = p$
  and $s_y^j(u) = u$, therefore the equality is satisfied by
  assumption.\\
Let $j > 0$ such as $s_y^j(u)$ is defined and suppose by recurrence
assumption that $f_y^{j-1}(p) = \delta(i,s_y^{j-1}(u))$. By definition
of $f_y$, $f_y(f_y^{j-1}(p))$ is the congruence class of the word
$s_y(s_y^{j-1}(u))$. Consequently, $f_y^j(p) = \delta(q_0,s_y^j(u))$,
which completes the recurrence and the proof.
\end{proof}
\begin{corollary}
  The terminal states of $\mathfrak{A}(y)$ are the states of the
  suffix path of last, SP(last).
\end{corollary}
\begin{proof}
  First, we prove that states of the path suffix are terminal. Let $p$
  be any state of \textit{SP(last)}. One has $p = f_y^j(last)$ for
  some $j \ge 0$. Because \textit{last}$=\delta(q_0,y)$, Proposition
  2.3 implies $p = \delta(q_0,s_y^j(y))$; and as $s_y^j(y)$ is a
  suffix of y, $p$ is a terminal state.\\
Conversely, let $p$ be a terminal state of $\mathfrak{A}(y)$. Let then
$u$ be a suffix of $y$ such that $p = \delta(q_0,u)$. Since $u$ is a
suffix of $y$, we can consider the greatest integer $j \ge 0$ for
which $|u| \le |s_y^j(y)|$. By Lemma 1.2 one obtains $u \equiv_y
s_y^j(y)$. Thus, $p = \delta(q_0, s_y^j(y))$ by definition of
$\mathfrak{A}(y)$. Therefore, Proposition 2.3 applied to $y$ implies
$p = f_y^j(\mathit{last})$, which proves that $p$ appears in
\textit{SP(last)}. This ends the proof.
\end{proof}
\section{On-line construction}
We present an on-line construction algorithm works in linear space
with an execution time $O(|y| \times \log(\text{Card}\mathcal{A}))$.
The algorithm treats the prefixes of $y$ from $\varepsilon$ to $y$
itself. At each stage, just after having treated prefix $w$, the
following information is available:
\begin{itemize}
\item The suffix automaton $\mathfrak{A}(w)$ with its transition
  function $\delta$.
\item The table $f$, defined on the states of $\mathfrak{A}(w)$, which
  implements the suffix function $f_w$.
\item The table $L$, defined on the states of $\mathfrak{A}(w)$, which
  implements the function length $lg_w$.
\item The state \textit{last}.
\end{itemize}
Terminal states of $\mathfrak{A}(w)$ are not explicitly marked, they
are given implicitly by the suffix path of \textit{last}. 
\begin{algorithmic}
  \Function{SuffixAutomaton}{$y,n$}
  \State $M \gets $ \Call{NewAutomaton}{$ $}
  \State $L[initial(M)] \gets 0$
  \State $last[M] \gets initial(M)$
  \For{each letter $a$ of $y$, sequentially}
  \State \Call{Extension}{$a$}
  \EndFor
  \State $p \gets last[M]$
  \Repeat
  \State $terminal(p) \gets $ TRUE
  \State $p \gets f[p]$
  \Until{$p$ is defined}
  \State \Return $M$
  \EndFunction
  \State
\end{algorithmic}
\newpage
\begin{algorithmic}
  \Function{Extension}{$a$}
  \State $new \gets$ \Call{NewState}{$ $}
  \State $L[new] \gets L[last[M]] + 1$
  \State $p \gets last[M]$
  \Repeat
  \State $adj[p] \gets adj[p] \cup \left\{(a, new)\right\}$
  \State $p \gets f[p]$
  \Until{$p$ is defined and \Call{Target}{$p,a$} is undefined}
  \If{$p$ is undefined}
  \State $f[new] \gets initial(M)$
  \Else
  \State $q \gets$ \Call{Target}{$p, a$}
  \If{$(p,a,q)$ is a solid edge, that is, $L[p] + 1 = L[q]$}
  \State $f[new] \gets q$
  \Else
  \State $f[new] \gets$ \Call{Clone}{$p,a,q$}
  \EndIf
  \EndIf
  \State $last[M] \gets new$
  \EndFunction
  \State
  \State
  \Function{Clone}{$p,a,q$}
  \State $clone \gets$ \Call{NewState}{$ $}
  \State $L[clone] \gets L[p] + 1$
  \State $adj[clone] \gets adj[q]$
  \State $f[clone] \gets f[q], f[q] \gets clone$
  \Repeat
  \State $adj[p] \gets adj[p] \backslash \left\{(a,q)\right\}$
  \State $adj[p] \gets adj[p] \cup \left\{(a,clone)\right\}$
  \State $p \gets f[p]$
  \Until{$p$ is defined and \Call{Target}{$p, a$}}
  \State \Return $clone$
  \EndFunction
\end{algorithmic}
\begin{theorem}
  Algorithm \textsc{SuffixAutomaton} builds a suffix automaton, that
  is \textsc{SuffixAutomaton}$(y)$ is the automaton $\mathfrak{y}$,
  for $y \in \mathcal{A}^*$.
\end{theorem}
When the first loop stops, three disjoint cases arise:
\begin{enumerate}[1.]
\item $p$ is not defined,
\item $(p,a,q)$ is a solid edge,
\item $(p,a,q)$ is a nonsolid edge.
\end{enumerate}
The following four figures are described those three difference cases.\\ 
\begin{figure}
\centering
\begin{tikzpicture}[->,shorten >=1pt,node distance=1.5cm,auto,line width=0.4mm]

  \tikzstyle{accepting}=[double distance=1.5pt]

  \node[state,initial,accepting] (0) {$0$};
  \node[state,accepting] (1) [right of=0] {$1$};
  \node[state,accepting] (2) [right of=1] {$2$};
  \node[state,accepting] (3) [right of=2] {$3$};
  \node[state] (4) [right of=3] {$4$};
  \node[state] (5) [right of=4] {$5$};
  \node[state] (6) [right of=5] {$6$};
  \node[state] (7) [right of=6] {$7$};
  \node[state] (8) [right of=7] {$8$};
  \node[state,accepting] (9) [right of=8] {$9$};
  \node[state] (5') [below =2.3cm of 5] {$5'$};
  \path[->]
  (0) edge node {$c$} (1)
  (1) edge node {$c$} (2)
  (2) edge node {$c$} (3)
  (3) edge node {$c$} (4)
  (4) edge node {$b$} (5)
  (5) edge node {$b$} (6)
  (6) edge node {$c$} (7)
  (7) edge node {$c$} (8)
  (8) edge node {$c$} (9)
  
  (1) edge [out=-70, in=-90] node [swap] {$b$} (5)
  (2) edge [out=-60, in=-100] node [swap] {$b$} (5)
  (3) edge [bend right=60] node [swap] {$b$} (5)
  (0) edge [bend right] node [swap] {$b$} (5')
  (5') edge [bend right] node [swap] {$b$} (6)
  (5') edge [bend right] node [swap] {$c$} (7)
  ;
  
\end{tikzpicture}
\caption{Suffix automaton $\mathfrak{A}(ccccbbccc)$ on which is illustrated
  in Figure \ref{fig:F2}, Figure \ref{fig:F3} and Figure \ref{fig:F4} the procedure \textsc{Extension}$(a)$ according to three
cases.}
\label{fig:F1}
\end{figure}

\begin{figure}
\centering
\begin{tikzpicture}[->,shorten >=1pt,node distance=1.4cm,auto,line width=0.4mm]

  \tikzstyle{accepting}=[double distance=1.5pt]

  \node[state,initial,accepting] (0) {$0$};
  \node[state] (1) [right of=0] {$1$};
  \node[state] (2) [right of=1] {$2$};
  \node[state] (3) [right of=2] {$3$};
  \node[state] (4) [right of=3] {$4$};
  \node[state] (5) [right of=4] {$5$};
  \node[state] (6) [right of=5] {$6$};
  \node[state] (7) [right of=6] {$7$};
  \node[state] (8) [right of=7] {$8$};
  \node[state] (9) [right of=8] {$9$};
  \node[state] (5') [below =2.3cm of 5] {$5'$};
  \node[state,accepting] (10) [right of=9] {$10$};
  \path[->]
  (0) edge node {$c$} (1)
  (1) edge node {$c$} (2)
  (2) edge node {$c$} (3)
  (3) edge node {$c$} (4)
  (4) edge node {$b$} (5)
  (5) edge node {$b$} (6)
  (6) edge node {$c$} (7)
  (7) edge node {$c$} (8)
  (8) edge node {$c$} (9)
  (9) edge node {$d$} (10)
  
  (1) edge [out=-70, in=-90] node [swap] {$b$} (5)
  (2) edge [out=-60, in=-100] node [swap] {$b$} (5)
  (3) edge [bend right=60] node [swap] {$b$} (5)
  (0) edge [bend right] node [swap] {$b$} (5')
  (5') edge [bend right] node [swap] {$b$} (6)
  (5') edge [bend right] node [swap] {$c$} (7)

  (0) edge [out=60, in=120] node {$d$} (10)
  (1) edge [out=50, in=130] node {$d$} (10)
  (2) edge [out=40, in=140] node {$d$} (10)
  (3) edge [out=30, in=150] node {$d$} (10)
  ;
  
\end{tikzpicture}
\caption{Suffix automaton $\mathfrak{A}(ccccbbcccd)$ obtained by
  extending $\mathfrak{A}(ccccbbccc)$ of Figure \ref{fig:F1} by letter
$d$. During the execution of the first loop of
\textsc{Extension}$(d)$, state $p$ traverses the suffix path $\langle
9, 3, 2, 1, 0 \rangle$ At the same time, edges labelled by letter $d$
are created, starting from these states and leading to $10$, the last
created state. The loop stops at the initial state. This situation
corresponds to Corollary 1.12.}
\label{fig:F2}
\end{figure}

\begin{figure}
\centering
\begin{tikzpicture}[->,shorten >=1pt,node distance=1.4cm,auto,line width=0.4mm]

  \tikzstyle{accepting}=[double distance=1.5pt]

  \node[state,initial,accepting] (0) {$0$};
  \node[state,accepting] (1) [right of=0] {$1$};
  \node[state,accepting] (2) [right of=1] {$2$};
  \node[state,accepting] (3) [right of=2] {$3$};
  \node[state,accepting] (4) [right of=3] {$4$};
  \node[state] (5) [right of=4] {$5$};
  \node[state] (6) [right of=5] {$6$};
  \node[state] (7) [right of=6] {$7$};
  \node[state] (8) [right of=7] {$8$};
  \node[state] (9) [right of=8] {$9$};
  \node[state] (5') [below =2.3cm of 5] {$5'$};
  \node[state,accepting] (10) [right of=9] {$10$};
  \path[->]
  (0) edge node {$c$} (1)
  (1) edge node {$c$} (2)
  (2) edge node {$c$} (3)
  (3) edge node {$c$} (4)
  (4) edge node {$b$} (5)
  (5) edge node {$b$} (6)
  (6) edge node {$c$} (7)
  (7) edge node {$c$} (8)
  (8) edge node {$c$} (9)
  (9) edge node {$c$} (10)
  
  (1) edge [out=-70, in=-90] node [swap] {$b$} (5)
  (2) edge [out=-60, in=-100] node [swap] {$b$} (5)
  (3) edge [bend right=60] node [swap] {$b$} (5)
  (0) edge [bend right] node [swap] {$b$} (5')
  (5') edge [bend right] node [swap] {$b$} (6)
  (5') edge [bend right] node [swap] {$c$} (7)
  ;
  
\end{tikzpicture}
\caption{Suffix automaton $\mathfrak{A}(ccccbbcccc)$ obtained by
  extending $\mathfrak{A}(ccccbbccc)$ of Figure \ref{fig:F1} by letter
$c$. The first loop of the procedure \textsc{Extension}$(c)$ stops at
state $3 = f[9]$ because an edge labelled by $c$ starts from this
state. Moreover, the edge $(3, c, 4)$ is solid. We obtain directly the
suffix link of the new state created: $f[10] = \delta(3,c) = 4$. There
is nothing else to do according to Corollary 1.11.}
\label{fig:F3}
\end{figure}

\begin{figure}
\centering
\begin{tikzpicture}[->,shorten >=1pt,node distance=1.4cm,auto,line width=0.4mm]

  \tikzstyle{accepting}=[double distance=1.5pt]

  \node[state,initial,accepting] (0) {$0$};
  \node[state] (1) [right of=0] {$1$};
  \node[state] (2) [right of=1] {$2$};
  \node[state] (3) [right of=2] {$3$};
  \node[state] (4) [right of=3] {$4$};
  \node[state] (5) [right of=4] {$5$};
  \node[state] (6) [right of=5] {$6$};
  \node[state] (7) [right of=6] {$7$};
  \node[state] (8) [right of=7] {$8$};
  \node[state] (9) [right of=8] {$9$};
  \node[state,accepting] (5') [below =2.3cm of 5] {$5'$};
  \node[state,accepting] (5'') [above =2.3cm of 5] {$5''$};
  \node[state,accepting] (10) [right of=9] {$10$};
  \path[->]
  (0) edge node {$c$} (1)
  (1) edge node {$c$} (2)
  (2) edge node {$c$} (3)
  (3) edge node {$c$} (4)
  (4) edge node {$b$} (5)
  (5) edge node {$b$} (6)
  (6) edge node {$c$} (7)
  (7) edge node {$c$} (8)
  (8) edge node {$c$} (9)
  (9) edge node {$b$} (10)
  
  (1) edge [bend left] node {$b$} (5'')
  (2) edge [bend left] node {$b$} (5'')
  (3) edge [bend left] node {$b$} (5'')
  (0) edge [bend right] node [swap] {$b$} (5')
  (5') edge [bend right] node [swap] {$b$} (6)
  (5') edge [bend right] node [swap] {$c$} (7)
  (5'') edge [bend left] node {$b$} (6)
  ;
  
\end{tikzpicture}
\caption{Suffix automaton $\mathfrak{A}(ccccbbcccb)$ obtained by
  extending $\mathfrak{A}(ccccbbccc)$ of Figure \ref{fig:F1} by letter
$b$. The first loop of the procedure \textsc{Extension}$(c)$ stops at
state $3 = f[9]$ because an edge labelled by $b$ starts from this
state. In the automaton $\mathfrak{A}(ccccbbccc)$ edge$(3,b,5)$ is not
solid. The word $cccb$ is a suffix of $ccccbbcccb$ but $ccccb$ is not,
although they both lead to state $5$. This state is duplicated into
the final state 5'' that is the class of factors $cccb, ccb$ and
$cb$. Edges $(3,b,5),(2,b,5)$ and $(1,b,5)$ of
$\mathfrak{A}(ccccbbccc)$ are redirected onto 5'' according to Theorem
1.10.}
\label{fig:F4}
\end{figure}

\end{document}